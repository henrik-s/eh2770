\section{Transformation Plan}
\label{sec:transformation_plan}
This section contains the identified gaps between As-Is and To-Be, these gaps can be divided up and translated into key work packages essential for realising the envisioned architecture. Additionally, the work packages are then depicted in a gantt diagram; the purpose of this is twofold, to estimate the duration of each work package as well as getting a sense for the internal dependencies among them. This section presents this whole transformation plan of how to reach the To-Be architecture from today's.
\begin{table}[H]
	\centering
	\begin{tabular}{|p{2.4cm}|l|l|p{7.5cm}|}
		\hline
		\textbf{} & \textbf{As-Is} & \textbf{To-Be} & \multicolumn{1}{c|}{\textbf{Gaps}} \\ \hline
		\textbf{Business architecture} &  &  & \begin{itemize}\vspace{-0.5cm}
		\item[$-$] Order manager
		\item[$-$] Send Order Acceptance
		\item[$+$] Present order completion
		\item[$+$] Instant feedback through "Applying order"
		\end{itemize}\vspace{-0.7cm} \\ \hline 

		\textbf{Information architecture} &  &  & \begin{itemize}\vspace{-0.5cm}
		\item[$-$] Customer order status
		\item[$-$] Order management system
		\item[$+$] Updated website including new application function
		\end{itemize}\vspace{-0.7cm} \\ \hline

		\textbf{Technology architecture} &  &  & \begin{itemize}\vspace{-0.5cm}
		\item[$-$] Fetch order information
		\item[$-$] Order data management
		\item[$-$] Order management DB
		\end{itemize}\vspace{-0.7cm} \\ \hline
	\end{tabular}	
	\caption{Order Process}
	\label{table:gaps_order}
\end{table}

\begin{table}[H]
	\centering
	\begin{tabular}{|p{2.4cm}|l|l|p{7.5cm}|}
		\hline
		\textbf{} & \textbf{As-Is} & \textbf{To-Be} & \multicolumn{1}{c|}{\textbf{Gaps}} \\ \hline
		\textbf{Business architecture} &  &  & \begin{itemize} \vspace{-0.5cm}
		\item[$-$] Claim administrator
		\item[$-$] Claim Application Download
		\item[$-$] Download Claim Form
		\end{itemize} \vspace{-0.7cm}\\ \hline 

		\textbf{Information architecture} &  & & \begin{itemize}\vspace{-0.5cm}
		\item[$-$] Form downloading
		\item[$-$] Form presentation
		\item[$+$] Customer Claim portal
		\item[$+$] Extendended website functionality
		\end{itemize}\vspace{-0.7cm} \\ \hline

		\textbf{Technology architecture} & - & - & \multicolumn{1}{c|}{\emph{No gaps}} \\ \hline
	\end{tabular}	
	\caption{Claim Process}
	\label{table:gaps_claim}
\end{table}

\begin{table}[H]
	\centering
	\begin{tabular}{|p{2.4cm}|l|l|p{7.5cm}|}
		\hline
		\textbf{} & \textbf{As-Is} & \textbf{To-Be} & \multicolumn{1}{c|}{\textbf{Gaps}} \\ \hline
		\textbf{Business architecture} &  &  & \begin{itemize}\vspace{-0.5cm}
		\item[$-$] Common support entry point "Support service"
		\item[$\pm$] Helpdesk employee replaces mail \& phone supporter
		\end{itemize}\vspace{-0.7cm} \\ \hline 

		\textbf{Information architecture} &  &  & \begin{itemize}\vspace{-0.5cm}
		\item[$-$] Phone support system
		\item[$-$] Mail support system
		\item[$-$] Mail recognition
		\item[$+$] Helpdesk system including application functions
		\item[$+$] Updated Mail Handler
		\end{itemize}\vspace{-0.7cm} \\ \hline

		\textbf{Technology architecture} &  &  & \begin{itemize}\vspace{-0.5cm}
		\item[$-$] Phone support DB
		\item[$-$] Mail support DB
		\item[$+$] Helpdesk DB
		\end{itemize}\vspace{-0.7cm} \\ \hline
	\end{tabular}	
	\caption{Support Process}
	\label{table:gaps_support}
\end{table}
%
\subsection{Activities}
In the following three tables the activities associated with respective model (claim, order and support) are listed. For each table the activities are ordered by execution priority, i.e. activities are executed top-down.
\begin{table}[H]
	\centering
	\begin{tabular}{|c|p{3cm}|p{10.5cm}|}
		\hline
		\textbf{\#} & \multicolumn{1}{c|}{\textbf{Role}} & \multicolumn{1}{c|}{\textbf{Activity}} \\ \hline
		\textbf{1} & Developers, Project managers, Business developers & Develop the new website functionality \\ \hline
		\textbf{2} & System administrator, developers & Integrate website with existing CRM DB\\ \hline
		\textbf{3} & Testers, System administrator  & Deploy beta version in parallel with existing system and review test run \\ \hline
		\textbf{4} & System administrator, developers & Deploy production version of the new functionality of the website \\ \hline
		\textbf{5} & Human Resources, System administrator & Replace the order manager with the automated order registration process. And remove Order management system\\ \hline

	\end{tabular}	
	\caption{"Order" activities}
	\label{table:activities_order}
\end{table}

\begin{table}[H]
	\centering
	\begin{tabular}{|c|p{3cm}|p{10.5cm}|}
		\hline
		\textbf{\#} & \multicolumn{1}{c|}{\textbf{Role}} & \multicolumn{1}{c|}{\textbf{Activity}} \\ \hline
		\textbf{1} & CIO, System administrator, Business developer & Analyze how to proceed with the website functionality extension. Who, what and when are questions answered here  \\ \hline
		\textbf{2} & Developers, Project manager(s) & Proceed with development of "Customer Claim Portal" and integrate this with Claim Management system\\ \hline
		\textbf{3} & Project manager, testers, business developers & Review system and ensure function and service compatibility \\ \hline
		\textbf{4} &System administrator, testers  & Deploy beta version to website and test run \\ \hline
		\textbf{5} & System administrators & Release the new Claim portal of the website\\ \hline
		\textbf{6} & HR, system administrators & Remove claim administrator as well as functions and services related to form downloading\\ \hline
	\end{tabular}	
	\caption{"Claim" activities}
	\label{table:activities_claim}
\end{table}

\begin{table}[H]
	\centering
	\begin{tabular}{|c|p{3cm}|p{10.5cm}|}
		\hline
		\textbf{\#} & \multicolumn{1}{c|}{\textbf{Role}} & \multicolumn{1}{c|}{\textbf{Activity}} \\ \hline
		\textbf{1} & Accountant, CIO, Project manager & Analyze benefits and drawbacks of developing or buying combined Helpdesk system \\ \hline
		\textbf{2} & Developers, Project manager & Procure combined helpdesk system\\ \hline
		\textbf{3} & Developers & Integrate helpdesk system, phone system dispatcher and mail server \\ \hline
		\textbf{4} & Business developers, Testers & Ensure that functionality and service support are maintained \\ \hline
		\textbf{5} & System administrator, Testers & Deploy support system in a test environment\\ \hline
		\textbf{6} & Developers, HR & Educate staff how to use the new system\\ \hline
		\textbf{7} & System administrator & Engage transition period by having both systems running in parallel\\ \hline
		\textbf{8} & System administrator & Remove old Phone Support System \& DB and Mail Support System \& DB\\ \hline
	\end{tabular}	
	\caption{"Support" activities}
	\label{table:activities_support}
\end{table}


\subsection{Gantt Scheme}
\begin{center}
	\begin{figure}[H]
		\centering
		\setlength\fboxsep{7pt}
		\setlength\fboxrule{0.5pt}
		\fbox{\includegraphics[scale = 0.5, angle=90]{images/gant_order.png}}
		\caption{Gantt chart for "Order" activities}
		\label{fig:gant_order}
	\end{figure}
\end{center}
\begin{center}
	\begin{figure}[H]
		\centering
		\setlength\fboxsep{7pt}
		\setlength\fboxrule{0.5pt}
		\fbox{\includegraphics[scale = 0.5, angle=90]{images/gant_claim.png}}
		\caption{Gantt chart for "Claim" activities}
		\label{fig:gant_claim}
	\end{figure}
\end{center}

\begin{center}
	\begin{figure}[H]
		\centering
		\setlength\fboxsep{7pt}
		\setlength\fboxrule{0.5pt}
		\fbox{\includegraphics[scale = 0.5, angle=90]{images/gant_support.png}}
		\caption{Gantt chart for "Support" activities}
		\label{fig:gant_support}
	\end{figure}
\end{center}